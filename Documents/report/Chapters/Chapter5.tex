
\chapter{Detailed System Design and Implementation}
\label{ch:implementation}

This chapter delves into the specific implementation details of the core components of the \textbf{ft\_transcendence} application, expanding on the architectural overview provided previously. It covers the backend logic, frontend structure, data persistence mechanisms, real-time communication, and the containerization strategy.

\section{Backend Implementation (Django)}
\label{sec:backend_impl}
The backend is built using the Django framework (version 5), leveraging its robust features for web development and the Django Rest Framework (DRF) for creating the RESTful API.

\subsection{Project Structure}
The Django project follows a standard layout, organized into distinct apps for modularity:
\begin{itemize}
    \item \texttt{users}: Handles user registration, authentication (native credentials, 42 OAuth), and basic profile information.
    \item \texttt{game}: Contains the logic for Pong gameplay, including game state management, rules enforcement, and matchmaking (for 1v1 online).
    \item \texttt{tournaments}: Manages the creation, progression, and recording of tournament brackets.
\end{itemize}

\subsection{API Design (DRF)}
The API provides endpoints for frontend interaction. DRF's ViewSets and Serializers are used to handle data validation, database interaction, and JSON response generation. Key endpoints likely include:
\begin{itemize}
    \item User authentication and registration.
    \item User profile information access.
    \item Game initiation and state updates.
    \item Tournament creation and status retrieval.
    \item Match history access.
\end{itemize}

\subsection{Authentication and Authorization}
Security is managed through multiple layers:
\begin{itemize}
    \item \textbf{Native Credentials:} Standard username/password login.
    \item \textbf{42 OAuth:} Integration with the 42 Intra's OAuth2 system for seamless login for 42 students.
    \item \textbf{JWT (JSON Web Tokens):} Simple-JWT library is used to issue JWTs stored in secure HTTP-only cookies upon successful login, authenticating subsequent API requests.
    \item \textbf{2FA (TOTP):} Optional Time-based One-Time Password (TOTP) using \texttt{django-otp} provides an additional security layer.
\end{itemize}

\section{Frontend Implementation (Vanilla JS)}
\label{sec:frontend_impl}
The frontend is a Single Page Application (SPA) built entirely with vanilla JavaScript (ES6+), HTML5, and styled with Bootstrap 5. This approach avoids reliance on heavy frontend frameworks.

\subsection{SPA Architecture}
JavaScript manages routing (likely using the History API or hash-based routing) to dynamically load content and views without full page reloads. State management might be handled through simple JavaScript objects or potentially a lightweight custom solution.

\subsection{Rendering}
\begin{itemize}
    \item \textbf{Game Interface:} The core Pong gameplay (paddles, ball, score) is rendered dynamically on an HTML5 \texttt{<canvas>} element, providing fine-grained control over animations and interactions.
    \item \textbf{UI Elements:} Other interface components (dashboard, login forms, profile page, menus) are built using standard HTML elements manipulated via the Document Object Model (DOM) by JavaScript. Bootstrap provides styling and layout components.
\end{itemize}

\subsection{API Interaction}
The frontend communicates with the Django backend via asynchronous JavaScript (e.g., using the \texttt{fetch} API) to send requests to the RESTful API endpoints and receive data in JSON format. Real-time updates (discussed below) are handled separately.

\section{Database Implementation}
\label{sec:database_impl}
Data persistence relies on two distinct systems:

\subsection{PostgreSQL}
A PostgreSQL (version 15) relational database serves as the primary datastore for user accounts, basic profile information, game settings, and detailed match history. Django's ORM (Object-Relational Mapper) facilitates interaction with the database.

\subsection{Ethereum Sidechain}
For tournament results, an Ethereum-compatible sidechain (potentially using Proof of Authority) is employed to ensure immutable and verifiable record-keeping. The Django backend interacts with a deployed smart contract via the \texttt{Web3.py} library to record the winners and potentially other key tournament data.

\section{Real-time Communication}
\label{sec:realtime_impl}
Real-time functionality, crucial for interactive gameplay, is implemented using \textbf{WebSockets}. Django Channels is the framework used within Django to handle WebSocket connections, allowing bidirectional communication between the server and connected clients for instant game state synchronization.

\section{DevOps and Containerization (Docker)}
\label{sec:devops_impl}
The entire application stack is containerized using Docker and orchestrated with Docker Compose, facilitating consistent development, testing, and deployment environments.

The \texttt{docker-compose.yml} file defines the services:
\begin{itemize}
    \item \textbf{Backend:} Runs the Django application (likely using Gunicorn or Uvicorn).
    \item \textbf{Frontend/Nginx:} An Nginx container serves the static frontend files (HTML, CSS, JS) and acts as a reverse proxy for the Django backend. It also handles TLS termination for HTTPS.
    \item \textbf{Database:} A PostgreSQL container manages the relational data.
    \item \textbf{ELK Stack:} Separate containers for Elasticsearch and Kibana handle centralized logging and monitoring.
\end{itemize}
Makefiles are likely used to streamline common Docker operations like building images, starting/stopping containers, and running management commands.