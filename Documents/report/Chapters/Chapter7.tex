\chapter{Blockchain Integration for Tournament Records}
\label{ch:blockchain}

An innovative feature of the \textbf{ft\_transcendence} project is the integration of blockchain technology to immutably record Pong tournament data. This chapter explains the implementation approach and how it enhances the application.

\section{Rationale for Blockchain Integration}
\label{sec:blockchain_rationale}
While the primary PostgreSQL database stores comprehensive match data, blockchain integration offers distinct advantages for tournament records:
\begin{itemize}
    \item \textbf{Immutability:} Tournament results become tamper-proof once recorded on the blockchain, providing a verifiable historical record.
    \item \textbf{Transparency:} Records can be independently verified by anyone with access to the blockchain network.
    \item \textbf{Educational Value:} Implementing blockchain serves as a technical showcase of modern decentralized technologies.
\end{itemize}

\section{Technology Implementation}
\label{sec:blockchain_tech}
\begin{itemize}
    \item \textbf{Ethereum Development Environment:} A local Ethereum blockchain is configured using a custom genesis block with appropriate settings for development purposes.
    \item \textbf{Web3.py Integration:} The Python Web3 library connects the Django backend to the Ethereum blockchain node, facilitating interaction with smart contracts.
    \item \textbf{Docker Integration:} The blockchain node runs in a dedicated Docker container within the application's overall containerized architecture.
\end{itemize}

\section{Tournament Data Structure}
\label{sec:data_structure}
The implementation uses two primary data structures:
\begin{itemize}
    \item \textbf{Match:} Represents individual match data with the following properties:
        \begin{itemize}
            \item Player1Name and Player2Name: Identifiers for match participants
            \item Player1Score and Player2Score: Final scores recorded for each player
            \item Winner: Identifier of the match winner
        \end{itemize}
    \item \textbf{Tournament:} Encapsulates a collection of matches with additional metadata:
        \begin{itemize}
            \item tournament\_name: Descriptive name of the tournament
            \item tournament\_id: Unique identifier assigned to the tournament
            \item matches: List of Match objects representing all games in the tournament
        \end{itemize}
\end{itemize}

\section{Backend Integration}
\label{sec:backend_integration}
The Django backend interacts with the blockchain through several key functions:
\begin{itemize}
    \item \textbf{init\_account():} Establishes connection to the blockchain node, sets up the default account, and verifies connectivity.
    \item \textbf{load\_contract():} Loads the deployed smart contract interface using its ABI and address from configuration files.
    \item \textbf{save\_tournament():} Records tournament data to the blockchain by calling the smart contract's createTournament function with tournament name and match data.
    \item \textbf{get\_tournaments():} Retrieves tournament records from the blockchain by name, returning structured data including match details and tournament metadata.
\end{itemize}

This integration follows a straightforward workflow:
\begin{enumerate}
    \item Blockchain connection is established using HTTP connection to the node running in Docker.
    \item When a tournament concludes, match details are formatted and submitted to the blockchain.
    \item Transaction receipt is awaited to ensure data is properly recorded.
    \item Tournament records can be retrieved by name for display in the application's user interface.
\end{enumerate}

\section{Security Considerations}
\label{sec:blockchain_security}
Several security aspects are addressed in the implementation:
\begin{itemize}
    \item \textbf{Account Management:} The application uses a predefined Ethereum account for blockchain interaction.
    \item \textbf{Network Isolation:} The blockchain node operates within the Docker network, limiting exposure to external access.
    \item \textbf{Configuration Security:} Blockchain configuration parameters are stored securely in dedicated configuration files.
\end{itemize}

\section{Limitations and Alternatives}
\label{sec:blockchain_limits}
\begin{itemize}
    \item \textbf{Development Focus:} The current implementation is primarily designed for educational and demonstration purposes rather than production-grade security.
    \item \textbf{Performance Considerations:} Blockchain transactions introduce latency compared to traditional database operations.
    \item \textbf{Simplified Model:} The implementation uses a straightforward data model optimized for the specific requirements of Pong tournaments.
    \item \textbf{Alternatives:} For production environments, alternatives might include distributed ledger technologies with better performance characteristics or cryptographically signed logs in conventional databases.
\end{itemize}

Despite these limitations, the blockchain integration successfully demonstrates how decentralized technologies can complement traditional databases to provide additional verification and immutability for critical application data.