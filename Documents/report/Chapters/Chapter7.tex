\chapter{Blockchain Integration for Tournament Records}
\label{ch:blockchain}

An innovative feature of the \textbf{ft\_transcendence} project is the integration of blockchain technology to immutably record the outcomes of official Pong tournaments. This chapter details the rationale, design, and implementation of this integration.

\section{Rationale for Blockchain Integration}
\label{sec:blockchain_rationale}
While the primary database (PostgreSQL) stores comprehensive match and tournament data, utilizing blockchain offers distinct advantages for tournament results:
\begin{itemize}
    \item \textbf{Immutability:} Once recorded on the blockchain, tournament results become extremely difficult to alter or delete, providing a tamper-proof historical record.
    \item \textbf{Verifiability:} Anyone with access to the blockchain network (even if it's a private or consortium sidechain) can potentially verify the recorded results independently, enhancing transparency.
    \item \textbf{Decentralization (Conceptual):} Although potentially deployed on a controlled sidechain, it introduces the concept of decentralized record-keeping, contrasting with the centralized nature of the primary database.
\end{itemize}
This serves as both a technical showcase and a method to guarantee the integrity of high-stakes tournament outcomes within the game's ecosystem.

\section{Technology Choices}
\label{sec:blockchain_tech}
\begin{itemize}
    \item \textbf{Ethereum Sidechain:} An Ethereum-compatible sidechain was chosen rather than the mainnet, likely due to cost (gas fees) and performance considerations. A Proof of Authority (PoA) consensus mechanism might be used for controlled environments, offering faster transaction times and lower energy consumption compared to Proof of Work.
    \item \textbf{Smart Contract (Solidity):} The logic for recording and potentially retrieving tournament data is encapsulated in a smart contract written in Solidity, the standard language for Ethereum-based development.
    \item \textbf{Web3.py:} The Django backend interacts with the deployed smart contract on the sidechain using the Python library \texttt{Web3.py}. This library allows the backend to connect to an Ethereum node, load the contract's Application Binary Interface (ABI), and call its functions.
\end{itemize}

\section{Smart Contract Design}
\label{sec:smart_contract}
The smart contract is designed to be simple yet effective for its specific purpose. Key aspects include:
\begin{itemize}
    \item \textbf{State Variables:} Stores essential information, likely mapping a tournament identifier to the winner's user ID and perhaps a timestamp. Example: \texttt{mapping(uint256 => TournamentResult) public tournamentResults;}
    \item \textbf{Structs:} A struct (e.g., \texttt{TournamentResult}) might be used to group related data: \texttt{struct TournamentResult { uint256 winnerUserId; uint256 timestamp; }}
    \item \textbf{Functions:}
        \begin{itemize}
            \item \texttt{recordTournament(uint256 tournamentId, uint256 winnerUserId)}: A function callable only by an authorized address (likely the backend server's wallet) to record the outcome of a completed tournament. It would populate the state variables.
            \item \texttt{getTournamentWinner(uint256 tournamentId) returns (uint256)}: A public view function to retrieve the winner of a specific tournament ID from the stored data.
        \end{itemize}
    \item \textbf{Events:} An event (e.g., \texttt{event TournamentRecorded(uint256 indexed tournamentId, uint256 indexed winnerUserId);}) is likely emitted when a tournament is recorded, allowing off-chain applications or indexers to easily track new records.
\end{itemize}

\section{Backend Interaction (Django)}
\label{sec:backend_interaction}
The Django backend orchestrates the interaction with the smart contract:
\begin{enumerate}
    \item \textbf{Connection Setup:} Using \texttt{Web3.py}, the backend connects to an accessible node of the Ethereum sidechain (specified via its RPC URL).
    \item \textbf{Contract Loading:} The backend loads the smart contract's ABI and address.
    \item \textbf{Transaction Trigger:} Upon the confirmed conclusion of a tournament within the application logic, the backend prepares to call the \texttt{recordTournament} function.
    \item \textbf{Signing and Sending:} The backend uses its configured private key to sign the transaction and sends it to the sidechain network via \texttt{Web3.py}. This requires the backend's wallet address to have sufficient funds (native sidechain currency) to cover any transaction gas fees, even if minimal on a PoA chain.
    \item \textbf{Confirmation Handling:} The backend might wait for transaction confirmation or handle it asynchronously, potentially updating the local PostgreSQL database status once the blockchain record is confirmed.
    \item \textbf{Reading Data (Optional):} The backend could also use \texttt{Web3.py} to call view functions like \texttt{getTournamentWinner} if needed, though primary data retrieval likely still relies on PostgreSQL for efficiency.
\end{enumerate}

\section{Security Considerations}
\label{sec:blockchain_security}
Implementing blockchain integration requires careful security management:
\begin{itemize}
    \item \textbf{Smart Contract Security:} The Solidity code needs auditing for common vulnerabilities (e.g., reentrancy, integer overflow/underflow), although the contract's simplicity reduces the attack surface. Access control on functions like \texttt{recordTournament} is critical, ensuring only the authorized backend wallet can call it.
    \item \textbf{Private Key Management:} The private key for the backend's wallet, used to sign transactions, must be stored securely (e.g., using environment variables or a secrets management system) and never exposed in the codebase. Compromise of this key would allow fraudulent tournament records.
    \item \textbf{Node Access:} Secure communication (e.g., HTTPS or WSS) with the sidechain node is necessary.
    \item \textbf{Gas Management:} While likely low on a sidechain, the backend must handle potential gas costs and ensure its wallet maintains a sufficient balance.
\end{itemize}

\section{Limitations and Alternatives}
\label{sec:blockchain_limits}
\begin{itemize}
    \item \textbf{Complexity:} Adds significant technical complexity compared to solely using PostgreSQL.
    \item \textbf{Cost:} While sidechains reduce costs, there are still infrastructure and maintenance overheads associated with running or connecting to the sidechain node.
    \item \textbf{Speed:} Blockchain transactions are inherently slower than direct database writes.
    \item \textbf{Alternatives:} A cryptographically signed log stored in a conventional database or distributed file system could offer some degree of tamper evidence without the full overhead of a blockchain.
\end{itemize}
Despite the limitations, the blockchain integration serves as a valuable demonstration of applying this technology to ensure the integrity of specific, high-value data points within the application.